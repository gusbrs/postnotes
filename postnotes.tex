% \iffalse meta-comment
%
% File: postnotes.tex
%
% This file is part of the LaTeX package "postnotes".
%
% Copyright (C) 2022  Gustavo Barros
%
% It may be distributed and/or modified under the conditions of the
% LaTeX Project Public License (LPPL), either version 1.3c of this
% license or (at your option) any later version.  The latest version
% of this license is in the file:
%
%    https://www.latex-project.org/lppl.txt
%
% and version 1.3 or later is part of all distributions of LaTeX
% version 2005/12/01 or later.
%
%
% This work is "maintained" (as per LPPL maintenance status) by
% Gustavo Barros.
%
% This work consists of the files postnotes.dtx,
%                                 postnotes.ins,
%                                 postnotes.tex,
%                                 postnotes-code.tex,
%         and the files listed in MANIFEST.md.
%
% The released version of this package is available from CTAN.
%
% -----------------------------------------------------------------------
%
% The development version of the package can be found at
%
%    https://github.com/gusbrs/postnotes
%
% for those people who are interested.
%
% -----------------------------------------------------------------------
%
% \fi

\documentclass{l3doc}

% The package itself *must* be loaded so that \GetFileInfo can pick up date
% and version data.
\usepackage{postnotes}
\postnotesetup{
  heading={
    \section*{\pntitle}
    \markright{\pnheaderdefault}
    \addcontentsline{toc}{section}{\pntitle}
  } ,
}

\usepackage[T1]{fontenc}

\usepackage[sc]{mathpazo}
\linespread{1.05}
\usepackage[scale=.88]{tgheros} % sans
\usepackage[varqu,scaled=1.03]{inconsolata} % tt
\usepackage{microtype}

\hypersetup{hidelinks}

\usepackage{zref-clever}
\zcsetup{
  cap,
  check,
  titleref,
  countertype = { pnexample = example } ,
}

\usepackage{listings}
\lstdefinestyle{code}{
  language=[LaTeX]TeX,
  moretexcs={
  }
}
\lstdefinestyle{postnotes}{
  style=code,
  moretexcs={
  }
}
\lstset{
  style=postnotes,
  basicstyle=\ttfamily\small,
  columns=fullflexible,
  keepspaces,
  xleftmargin=\leftmargin,
  xrightmargin=.5\leftmargin,
}
% Setup inspired by https://tex.stackexchange.com/a/4068. For how to use these
% environments in a .dtx context see https://tex.stackexchange.com/a/31026.
\newcounter{pnexample}
\lstnewenvironment{pnexample}[1][]{%
  \renewcommand{\lstlistingname}{Example}%
  \renewcommand*\theHlstlisting{ht.\thelstlisting}%
  \lstset{#1}%
  \setcounter{lstlisting}{\value{pnexample}}%
}{}
\lstnewenvironment{pnsnippet}[1][]{%
  \renewcommand{\lstlistingname}{Example}%
  \lstset{#1}%
}{}
\ExplSyntaxOn
\makeatletter
\lst@AddToHook { PreInit }
  {
    \cs_if_exist:cT { c@ \lstenv@name }
      { \exp_args:Nx \refstepcounter { \lstenv@name } }
  }
\makeatother
\ExplSyntaxOff

\NewDocumentCommand\opt{m}{\texttt{#1}}

\begin{document}

\GetFileInfo{postnotes.sty}

\title{%
  The \pkg{postnotes} package%
  \thanks{This file describes \fileversion, released \filedate.}%
  \texorpdfstring{\\{}\medskip{}}{ - }%
  User manual%
  \texorpdfstring{\medskip{}}{}%
}

\author{%
  Gustavo Barros%
  \thanks{\url{https://github.com/gusbrs/postnotes}}%
}

\date{\filedate}

\maketitle

\begin{center}
  {\bfseries \abstractname\vspace{-.5em}\vspace{0pt}}
\end{center}

\begin{quotation}
  \pkg{postnotes} is an endnotes package for \LaTeX{}.  Its user interface
  provides means to print multiple sections of notes along the document, and
  to subdivide them either automatically -- by chapter, by section -- or at
  manually specified places, thus being able to easily handle both numbered
  and unnumbered headings.  The package also provides infrastructure for
  setting up contextual running headers for printed notes.  The default is a
  simple but useful one, in the form ``Notes to pages N--M'', but more
  elaborate ones can be built.  When \pkg{hyperref} is loaded, \pkg{postnotes}
  provides hyperlinked notes, including back links.
\end{quotation}

\clearpage{}

\tableofcontents

\clearpage{}

\section{Introduction}

\pkg{postnotes} is an endnotes package for \LaTeX{}.  Its user interface
provides means to print multiple sections of notes along the document, and to
subdivide them either automatically -- by chapter, by section -- or at
manually specified places, thus being able to easily handle both numbered and
unnumbered headings.  The package also provides infrastructure for setting up
contextual running headers for printed notes.  The default is a simple but
useful one, in the form ``Notes to pages N--M'', but more elaborate ones can
be built.  When \pkg{hyperref} is loaded, \pkg{postnotes} provides hyperlinked
notes, including back links.

Though this feature set is mostly (albeit not completely) available in one or
another of the existing endnotes packages for \LaTeX{}, subsets of it exist in
individual packages, not necessarily compatible with each other.
\pkg{postnotes} brings these features together in one place, with no external
dependencies except an up-to-date kernel.

On the technical side, \pkg{postnotes} is peculiar among existing \LaTeX{}
packages in this area of functionality by the fact that it does not use an
external file to store the notes.  Both the notes' contents and its metadata
are stored in variables which are later retrieved at the time of printing.  In
particular, the content of the note is stored and retrieved with ``no
manipulation'' (as in \texttt{expl3}'s \texttt{N}/\texttt{n} function
signatures) and only gets to be expanded at the time it is meant to be
typeset.  The \file{.aux} file is leveraged to set page labels for the notes,
since that particular information has to be retrieved asynchronously but,
other than that, variables are employed to pass information around.

This has some advantages.  First, as is well known, sending arbitrary content
to a file to be read later is not a noiseless process in \LaTeX{}.  Thus, not
doing so makes things smoother.  Second, the external file approach is
strictly linear: the notes which were written to the file get printed as such,
in the order they were written.  Having the notes available as a set of
variables allows for some more flexibility than that, through the possibility
of pre-processing the notes before printing.  It also brings some extra
degrees of freedom in storing note metadata, and in restoring part of the
environment where the note is called to where the note's content is printed.

\section{Loading the package}

\pkg{postnotes} can be loaded with the usual:

\begin{pnsnippet}
\usepackage{postnotes}
\end{pnsnippet}

The package does not accept load-time options, package options must be set
using \cs{postnotesetup} (see \zcref{sec:options},
\zcref[ref=title,noname]{sec:package-options}).

\section{User interface}

\begin{function}{\postnote}
  \begin{syntax}
    \cs{postnote}\oarg{options}\marg{text}
  \end{syntax}
\end{function}
Sets a postnote with content \meta{text}.  A note ``mark'' is typeset at the
place \cs{postnote} is called, and \meta{text} is stored to be typeset later,
on the next call to \cs{printpostnotes}.  The mark is usually determined by
the printed representation of the main counter, \texttt{postnote}, but can be
manually set too.  \cs{postnote} can receive a number of \meta{options}, which
are presented in \zcref{sec:options},
\zcref[ref=title,noname]{sec:note-options}.

\begin{function}{\postnotesection}
  \begin{syntax}
    \cs{postnotesection}\oarg{options}\marg{text}
  \end{syntax}
\end{function}
Sets a postnote section with content \meta{text}.  This is the basic interface
for subdividing the notes when printed.  For those familiar with it, this
command is \pkg{postnotes}'s equivalent to \pkg{endnotes}' \cs{addtoendnotes}.
It is also intended to add text or commands along the notes' sequence at the
point it is called, but \cs{postnotesection} behaves somewhat differently,
prominently it is skipped at \cs{printpostnotes} if a section contains no
notes.  In other words, if two (or more) calls of \cs{postnotesection} occur
in immediate sequence, with no \cs{postnote} in between, the latter one takes
precedence over the former, instead of being accumulated in the queue.  This
is intended to facilitate the automation of the subdivision of the notes.  So,
one can, for example, use a hook to \cs{chapter} and not have to worry if a
chapter with no notes will generate an empty section inside
\cs{printpostnotes}, e.g., by the call to \cs{chapter*} at the table of
contents, and so on.  Or, one can use the heading number for the automated
case, but be able to override it manually for an occasional unnumbered one.
Hence, a more semantic name was chosen, instead of the generic ``add to''.
Its \meta{options} are presented in \zcref{sec:options},
\zcref[ref=title,noname]{sec:section-options}.

\begin{function}{\printpostnotes}
  \begin{syntax}
    \cs{printpostnotes}\oarg{options}
  \end{syntax}
\end{function}
Prints the \cs{postnotes} set since the last call of \cs{printpostnotes}, or
since the beginning of the document. Its \meta{options} are presented in
\zcref{sec:options}, \zcref[ref=title,noname]{sec:print-options}.

\begin{function}{\postnoteref}
  \begin{syntax}
    \cs{printpostnotes}\meta{*}\marg{label}
  \end{syntax}
\end{function}
Typesets a postnote reference to \meta{label}.  Of course, \meta{label} must
have been set to a particular postnote, which can be done by the standard
\cs{label} command.  The starred version of the command inhibits hyperlinking.
When the \pkg{zref-user} package is loaded, a corresponding \cs{postnotezref}
is also provided.


\section{Options}
\zlabel{sec:options}

\subsection*{Package options}
\zlabel{sec:package-options}

\subsection*{Note options}
\zlabel{sec:note-options}

\subsection*{Section options}
\zlabel{sec:section-options}

\subsection*{Print options}
\zlabel{sec:print-options}


\section{Notes sections}


\section{Headers}


\section{Cross-referencing}

% Basically, the difference between \label and label=.


\section{Acknowledgments}

Some people have kindly contributed to \pkg{postnotes}, whether they are aware
of it or not.  Suggestions, ideas, solutions to problems, bug reports or even
encouragement were generously provided by (in chronological order):
  Ulrike Fischer,
  % 2022-03-22: https://chat.stackexchange.com/transcript/message/60708390#60708390
  % 2022-03-28: https://chat.stackexchange.com/transcript/message/60754383#60754383
  % 2022-03-31: https://github.com/latex3/hyperref/issues/230
  % 2022-04-09: https://github.com/latex3/hyperref/issues/229
  David Carlisle,
  % 2022-03-28: https://chat.stackexchange.com/transcript/message/60754383#60754383
  % 2022-04-08: https://tex.stackexchange.com/a/640035 (comments)
  and Moritz Wemheuer.
  % 2022-04-05: https://tex.stackexchange.com/q/597359#comment1594585_597389

If I have inadvertently left anyone off the list I apologize, and please let
me know, so that I can correct the oversight.

Thank you all very much!


\section{Change history}

A change log with relevant changes for each version, eventual upgrade
instructions, and upcoming changes, is maintained in the package's repository,
at \url{https://github.com/gusbrs/postnotes/blob/main/CHANGELOG.md}.  An
archive of historical versions of the package is also kept at
\url{https://github.com/gusbrs/postnotes/releases}.


\end{document}
